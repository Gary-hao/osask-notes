\chapter{	FIFO与鼠标控制	}
\section{	获取按键编码(harib04a)	}
现在,只要在键盘上按一按键,就会在屏幕上显示信息,其他的我们什么也做不了。我们将程序改善一下,让程序在按下一个键后不会结束,而是把按键的编码在画面上显示出来,这样就可以切实完成中断处理程序了。

更改的程序是init.c程序中的inthandler21函数,具体如下:
\begin{code}
#define PORT_KEYDAT		0x0060

void inthandler21(int *esp)
{
	struct BOOTINFO *binfo = (struct BOOTINFO *) ADR_BOOTINFO;
	unsigned char data, s[4];
	io_out8(PIC0_OCW2, 0x61);	/* IRQ-01受付完了をPICに通知 */
	data = io_in8(PORT_KEYDAT);

	sprintf(s, "%02X", data);
	boxfill8(binfo->vram, binfo->scrnx, COL8_008484, 0, 16, 15, 31);
	putfonts8_asc(binfo->vram, binfo->scrnx, 0, 16, COL8_FFFFFF, s);

	return;
}
\end{code}

\cs

程序中|io_out8(PIC0_OCW2, 0x61);|这句话用来通知PIC已经知道发生了IRQ1 中断。如果是IRQ3,则写成0x63。执行这句话之后,PIC继续时刻监视IRQ1中断是否发生,反过来,如果忘记了执行这句话,PIC就不再监视IRQ1中断,不管下次由键盘输入什么信息,系统都感知不到了。

\cs
程序所完成的,是将接收到的按键编码显示在画面上,然后结束中断处理。

\section{	加快中断处理(harib04b)	}
程序里有一个问题,那就是字符显示的内容被放在了中断处理程序中。

所谓中断处理,基本上就是打断CPU本来的工作,加塞要求进行处理,所以必须完成得干净利索。而且中断处理进行期间,不再接收别的中断。所以如果我们处理键盘的中断速度太慢,就会出现鼠标的运动不连贯、不能从网上接收数据等情况。

另一方面,字符显示要花大块的时间来进行处理。仅仅画一个字符,就要执行8 $\times$16=128次if语句,来判断是否要往VRAM里描画该像素。如果判定为描画该像素,还要执行内存写入指令。而且为确定具体往内存的哪个地方写,还要做很多地址计算。这些事情,在我们看来,或许只是一瞬间的事情,但在计算机看来,可不是这样。

谁也不知道其他中断会在哪个瞬间到来。事实上,很可能在键盘输入的同时,就有数据正在从网上下载,而PIC在等待键盘中断处理的结束。

\cs

解决方案是先将按键的编码接收下来,保存到变量里去,然后由HariMain偶尔去看看这个变量。如果发现有了数据,就把它显示出来。

\begin{code}
struct KEYBUF keybuf;

void inthandler21(int *esp)
{
	unsigned char data;
	io_out8(PIC0_OCW2, 0x61);	/* IRQ-01受付完了をPICに通知 */
	data = io_in8(PORT_KEYDAT);
	if (keybuf.flag == 0) {
		keybuf.data = data;
		keybuf.flag = 1;
	}
	return;
}
\end{code}

考虑到键盘的输入时需要缓冲区,先定义一个构造体,命名为keybuf。其中的flag变量用于表示这个缓冲区是否为空。如果flag是0,表示缓冲区为空;如果flag为1,表示缓冲区中有数据。那么,如果缓冲区有数据,而这时又来了一个中断,那么该怎么办呢?先不管哈~

\cs

\begin{code}[label=bootpack.c中HariMain函数节选]
for (;;) {
		io_cli();
		if (keybuf.flag == 0) {
			io_stihlt();
		} else {
			i = keybuf.data;
			keybuf.flag = 0;
			io_sti();
			sprintf(s, "%02X", i);
			boxfill8(binfo->vram, binfo->scrnx, COL8_008484, 0, 16, 15, 31);
			putfonts8_asc(binfo->vram, binfo->scrnx, 0, 16, COL8_FFFFFF, s);
		}
	}
\end{code}
开始先用|io_cli|指令屏蔽中断。

如果flag的值是0,说明键还没有被按下,keybuf.data里没有值 保存下来。在keybuf.data里有值被保存下来之前我们无事可做,所以干脆去执行|io_hlt|。但是,由于已经执行了|io_cli|屏蔽了中断,如果这样就去执行HLT指令的话,即使没有什么键被按下,程序也不会有任何反应。所以STI和HLT都要执行,而执行这两个指令的函数就是|io_stihlt|。执行HLT指令以后,如果收到了PIC的通知,CPU就会被唤醒。这样,CPU首先会去执行中断处理程序。中断处理程序执行完之后,又回到for语句的开头,再执行|io_cli|函数。

如果通过中断处理函数在keybuf.data里存入了按键编码,else语句就会被执行。先将这个键码(keybuf.data)值保存到变量i里,然后将flag置为0表示键码值清为空,最后再通过|io_sti|语句开放中断。

\cs

运行程序,能够顺利执行……但是,右Ctrl键的显示是有问题的。

查阅资料得知,当按下右Ctrl键时,会产生两个字节的键码值“E0 1D”,而松开这个键之后,会产生两个字节的键码值“E0 9D”。在一次产生两个字节键码值的情况下,因为键盘内部电路一次只能发送一个字节,所以一次按键会产生两次中断,第一次中断时发送E0,第二次中断发生1D。

在harib04a中,以上两次中断所发送的值都能收到,瞬间显示E0后,紧接着又显示1D或者9D。而在harib04b中,HariMain函数在收到E0之前,又收到前一次按键产生的1D或者9D,而这个字节被舍弃了。

\section{	制作FIFO缓冲区(harib04c)	}
\section{	改善FIFO缓冲区(harib04d)	}
\section{	整理FIFO缓冲区(harib04e)	}
\section{	总算讲到鼠标了(harib04f)	}
\section{	从鼠标接受数据(harib04g)	}

