\chapter{	FIFO与鼠标控制	}
\section{	获取按键编码(harib04a)	}
现在,只要在键盘上按一按键,就会在屏幕上显示信息,其他的我们什么也做不了。我们将程序改善一下,让程序在按下一个键后不会结束,而是把按键的编码在画面上显示出来,这样就可以切实完成中断处理程序了。

更改的程序是init.c程序中的inthandler21函数,具体如下:
\begin{code}
#define PORT_KEYDAT		0x0060

void inthandler21(int *esp)
{
	struct BOOTINFO *binfo = (struct BOOTINFO *) ADR_BOOTINFO;
	unsigned char data, s[4];
	io_out8(PIC0_OCW2, 0x61);	/* IRQ-01受付完了をPICに通知 */
	data = io_in8(PORT_KEYDAT);

	sprintf(s, "%02X", data);
	boxfill8(binfo->vram, binfo->scrnx, COL8_008484, 0, 16, 15, 31);
	putfonts8_asc(binfo->vram, binfo->scrnx, 0, 16, COL8_FFFFFF, s);

	return;
}
\end{code}

\cs

程序中|io_out8(PIC0_OCW2, 0x61);|这句话用来通知PIC已经知道发生了IRQ1 中断。如果是IRQ3,则写成0x63。执行这句话之后,PIC继续时刻监视IRQ1中断是否发生,反过来,如果忘记了执行这句话,PIC就不再监视IRQ1中断,不管下次由键盘输入什么信息,系统都感知不到了。

\cs
程序所完成的,是将接收到的按键编码显示在画面上,然后结束中断处理。

\section{	加快中断处理(harib04b)	}
程序里有一个问题,那就是字符显示的内容被放在了中断处理程序中。

所谓中断处理,基本上就是打断CPU本来的工作,加塞要求进行处理,所以必须完成得干净利索。而且中断处理进行期间,不再接收别的中断。所以如果我们处理键盘的中断速度太慢,就会出现鼠标的运动不连贯、不能从网上接收数据等情况。

另一方面,字符显示要花大块的时间来进行处理。仅仅画一个字符,就要执行8 $\times$16=128次if语句,来判断是否要往VRAM里描画该像素。如果判定为描画该像素,还要执行内存写入指令。而且为确定具体往内存的哪个地方写,还要做很多地址计算。这些事情,在我们看来,或许只是一瞬间的事情,但在计算机看来,可不是这样。

谁也不知道其他中断会在哪个瞬间到来。事实上,很可能在键盘输入的同时,就有数据正在从网上下载,而PIC在等待键盘中断处理的结束。

\cs

解决方案是先将按键的编码接收下来,保存到变量里去,然后由HariMain偶尔去看看这个变量。如果发现有了数据,就把它显示出来。

\begin{code}
struct KEYBUF keybuf;

void inthandler21(int *esp)
{
	unsigned char data;
	io_out8(PIC0_OCW2, 0x61);	/* IRQ-01受付完了をPICに通知 */
	data = io_in8(PORT_KEYDAT);
	if (keybuf.flag == 0) {
		keybuf.data = data;
		keybuf.flag = 1;
	}
	return;
}
\end{code}

考虑到键盘的输入时需要缓冲区,先定义一个构造体,命名为keybuf。其中的flag变量用于表示这个缓冲区是否为空。如果flag是0,表示缓冲区为空;如果flag为1,表示缓冲区中有数据。那么,如果缓冲区有数据,而这时又来了一个中断,那么该怎么办呢?先不管哈~

\cs

\begin{code}[label=bootpack.c中HariMain函数节选]
for (;;) {
		io_cli();
		if (keybuf.flag == 0) {
			io_stihlt();
		} else {
			i = keybuf.data;
			keybuf.flag = 0;
			io_sti();
			sprintf(s, "%02X", i);
			boxfill8(binfo->vram, binfo->scrnx, COL8_008484, 0, 16, 15, 31);
			putfonts8_asc(binfo->vram, binfo->scrnx, 0, 16, COL8_FFFFFF, s);
		}
	}
\end{code}

开始先用|io_cli|指令屏蔽中断。

如果flag的值是0,说明键还没有被按下,keybuf.data里没有值 保存下来。在keybuf.data里有值被保存下来之前我们无事可做,所以干脆去执行|io_hlt|。但是,由于已经执行了|io_cli|屏蔽了中断,如果这样就去执行HLT指令的话,即使没有什么键被按下,程序也不会有任何反应。所以STI和HLT都要执行,而执行这两个指令的函数就是|io_stihlt|。执行HLT指令以后,如果收到了PIC的通知,CPU就会被唤醒。这样,CPU首先会去执行中断处理程序。中断处理程序执行完之后,又回到for语句的开头,再执行|io_cli|函数。

如果通过中断处理函数在keybuf.data里存入了按键编码,else语句就会被执行。先将这个键码(keybuf.data)值保存到变量i里,然后将flag置为0表示键码值清为空,最后再通过|io_sti|语句开放中断。

\cs

运行程序,能够顺利执行……但是,右Ctrl键的显示是有问题的。

查阅资料得知,当按下右Ctrl键时,会产生两个字节的键码值“E0 1D”,而松开这个键之后,会产生两个字节的键码值“E0 9D”。在一次产生两个字节键码值的情况下,因为键盘内部电路一次只能发送一个字节,所以一次按键会产生两次中断,第一次中断时发送E0,第二次中断发生1D。

在harib04a中,以上两次中断所发送的值都能收到,瞬间显示E0后,紧接着又显示1D或者9D。而在harib04b中,HariMain函数在收到E0之前,又收到前一次按键产生的1D或者9D,而这个字节被舍弃了。

\section{	制作FIFO缓冲区(harib04c)	}
问题在于这里创建的缓冲区只存储一个字节,如果做一个能够存储多字节的缓冲区,那么它就不会满,问题也就解决了。

根据这种思路,有一下程序:
\begin{code}
struct KEYBUF {
	unsigned char data[32];
	int next;
};

void inthandler21(int *esp)
{
	unsigned char data;
	io_out8(PIC0_OCW2, 0x61);	/* IRQ-01受付完了をPICに通知 */
	data = io_in8(PORT_KEYDAT);
	if (keybuf.next < 32) {
		keybuf.data[keybuf.next] = data;
		keybuf.next++;
	}
	return;
}
\end{code}

keybuf.next的起点是“0”,所以最初存储的数据是keybuf.data[0],共32个存储位置。

下一个存储位置用变量next来管理。这样就可以记住32个数据,而不会溢出,但是为保险起见,next的值变成32之后,就舍去不要了。

\cs

取得数据的程序如下:

\begin{code}
	for (;;) {
		io_cli();
		if (keybuf.next == 0) {
			io_stihlt();
		} else {
			i = keybuf.data[0];
			keybuf.next--;
			for (j = 0; j < keybuf.next; j++) {
				keybuf.data[j] = keybuf.data[j + 1];
			}
			io_sti();
			sprintf(s, "%02X", i);
			boxfill8(binfo->vram, binfo->scrnx, COL8_008484, 0, 16, 15, 31);
			putfonts8_asc(binfo->vram, binfo->scrnx, 0, 16, COL8_FFFFFF, s);
		}
	}
\end{code}

如果next不是0,则说明至少有一个数据。最开始的一个数据肯定是放在|data[0]|中的,将这个数据存入到变量i中去。这样,数就减少一个,所以将next减去1。

接下来,for循环中,数据的存放位置全部都向前移送一个位置。

\cs
此时,右Ctrl的处理运行正常。但从|data[0]|取得数据后有关数据移送的处理不尽如人意。

数据移送处理本身没有什么不好,只是在禁止中断期间做数据移送处理有问题。但如果在数据移送处理前就允许中断的话,会搞乱要处理的数据,这当然不行。下面解决。
\section{	改善FIFO缓冲区(harib04d)	}
想开发一个不需要数据移送操作的FIFO型缓冲区。基本思路是:不仅维护下一个要写入数据的位置,还要维护下一个要读出数据的位置。这就像数据读出位置在追着数据写入位置跑一样。这样就不需要数据移送操作了。数据读出位置追上数据写入位置的时候,就相当于缓冲区为空,没有数据。

但是这样的缓冲区使用一段时间后,下一个数据写入位置会变成31,而这时下一个数据读出位置可能已经是29或30什么的了。当下一个写入位置变成32的时候,就走到死胡同了。因为下面没地方可以写入数据了。

如果当下一个数据写入位置到达缓冲区终点时,数据读出位置也恰好到达缓冲区终点,也就是说缓冲区正好变空,那还好说。我们只要将下一个数据写入位置和下一个数据读出位置都再置为0就行了,就像转回去从头再来一样。

但是总还是会有数据读出位置没有追上数据写入位置的情况。这时,又不得不进行数据移送操作。原来是每次都要进行数据移送,而现在不用每次都做。

仔细想一下,当下一个数据写入位置到达缓冲区最末尾,缓冲区开头部分应该已经变空了(如果还没有变空,说明数据读出跟不上数据写入,只能把部分数据扔掉了)。因此如果下一个数据写入位置到了32以后,就强制性地将它设置为0.这样一来,下一个数据写入位置就跑到了下一个数据读出位置的后面,让人觉得怪怪的。但这无关紧要,没什么问题。

对下一个数据读出位置也做同样的处理,一旦到了32以后,就把它设置为从0开始继续读取数据。这样32字节的缓冲区就能一圈一圈地不停循环,长久使用。数据移送操作一次都不需要。

\cs

相应的代码如下:
\begin{code}[label=bootpack.h节选]
struct KEYBUF {
	unsigned char data[32];
	int next_r, next_w, len;
};
\end{code}

变量len是指缓冲区能记录多少字节的数据。

\begin{code}[label=int.c节选]
void inthandler21(int *esp)
{
	unsigned char data;
	io_out8(PIC0_OCW2, 0x61);	/* IRQ-01受付完了をPICに通知 */
	data = io_in8(PORT_KEYDAT);
	if (keybuf.len < 32) {
		keybuf.data[keybuf.next_w] = data;
		keybuf.len++;
		keybuf.next_w++;
		if (keybuf.next_w == 32) {
			keybuf.next_w = 0;
		}
	}
	return;
}
\end{code}

读出数据程序如下:
\begin{code}
	for (;;) {
		io_cli();
		if (keybuf.len == 0) {
			io_stihlt();
		} else {
			i = keybuf.data[keybuf.next_r];
			keybuf.len--;
			keybuf.next_r++;
			if (keybuf.next_r == 32) {
				keybuf.next_r = 0;
			}
			io_sti();
			sprintf(s, "%02X", i);
			boxfill8(binfo->vram, binfo->scrnx, COL8_008484, 0, 16, 15, 31);
			putfonts8_asc(binfo->vram, binfo->scrnx, 0, 16, COL8_FFFFFF, s);
		}
	}
\end{code}
\section{	整理FIFO缓冲区(harib04e)	}
将结构做成这样:
\begin{code}
struct FIFO8 {
	unsigned char *buf;
	int p, q, size, free, flags;
};
\end{code}

如果我们将缓冲区大小固定成32字节的话,以后改起来就不方便了,所以把它定义成可变的。缓冲区的总字节数保存在变量size里。变量free用于保存缓冲区里没有数据的字节数。缓冲区地址保存在变量buf里。p代表下一个数据写入地址,q代表下一个数据读出地址。
\begin{code}
void fifo8_init(struct FIFO8 *fifo, int size, unsigned char *buf)
/* FIFOバッファの初期化 */
{
	fifo->size = size;
	fifo->buf = buf;
	fifo->free = size; /* 空き */
	fifo->flags = 0;
	fifo->p = 0; /* 書き込み位置 */
	fifo->q = 0; /* 読み込み位置 */
	return;
}
\end{code}

|fifo8_init|是结构的初始化函数,用来设定各种初始值,也就是设定FIFO8结构的地址以及与结构有关的各种参数。

\begin{code}
#define FLAGS_OVERRUN		0x0001

int fifo8_put(struct FIFO8 *fifo, unsigned char data)
/* FIFOへデータを送り込んで蓄える */
{
	if (fifo->free == 0) {
		/* 空きがなくてあふれた */
		fifo->flags |= FLAGS_OVERRUN;
		return -1;
	}
	fifo->buf[fifo->p] = data;
	fifo->p++;
	if (fifo->p == fifo->size) {
		fifo->p = 0;
	}
	fifo->free--;
	return 0;
}
\end{code}

|fifo8_put|是往FIFO缓冲区存储1字节信息的函数。用flags这一变量来记录是否溢出。

\begin{code}
int fifo8_get(struct FIFO8 *fifo)
/* FIFOからデータを一つとってくる */
{
	int data;
	if (fifo->free == fifo->size) {
		/* バッファが空っぽのときは、とりあえず-1が返される */
		return -1;
	}
	data = fifo->buf[fifo->q];
	fifo->q++;
	if (fifo->q == fifo->size) {
		fifo->q = 0;
	}
	fifo->free++;
	return data;
}
\end{code}

|fifo8_get|是从FIFO缓冲区取出1字节的函数。
\begin{code}
int fifo8_status(struct FIFO8 *fifo)
/* どのくらいデータが溜まっているかを報告する */
{
	return fifo->size - fifo->free;
}
\end{code}

|fifo8_status|用来查看缓冲区状态。

使用以上函数,写成的程序段如下:
\begin{code}
struct FIFO8 keyfifo;

void inthandler21(int *esp)
{
	unsigned char data;
	io_out8(PIC0_OCW2, 0x61);	/* IRQ-01受付完了をPICに通知 */
	data = io_in8(PORT_KEYDAT);
	fifo8_put(&keyfifo, data);
	return;
}
\end{code}

MariMain函数内容如下:
\begin{code}
	char s[40], mcursor[256], keybuf[32];

	fifo8_init(&keyfifo, 32, keybuf);

	for (;;) {
		io_cli();
		if (fifo8_status(&keyfifo) == 0) {
			io_stihlt();
		} else {
			i = fifo8_get(&keyfifo);
			io_sti();
			sprintf(s, "%02X", i);
			boxfill8(binfo->vram, binfo->scrnx, COL8_008484, 0, 16, 15, 31);
			putfonts8_asc(binfo->vram, binfo->scrnx, 0, 16, COL8_FFFFFF, s);
		}
	}
\end{code}

程序运行正常!
\section{	总算讲到鼠标了(harib04f)	}
\section{	从鼠标接受数据(harib04g)	}

