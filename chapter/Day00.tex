\chapter{着手开发之前}

\section{	前言	}
\begin{quote}
阅读本书几乎不需要相关储备知识,这一点稍后还会详述。不管是用什么编程语言,只要是曾经写过简单的程序,对编程有一些感觉,就已经足够了(即使没有任何编程经验,应该也能看懂),因为这本书主要就是面向初学者的。书中虽然有很多C语言程序,但实际上并没有用到很高深的C语言知识,所以就算是曾经因为C语言太难而中途放弃的人也不用担心看不懂。当然,如果具备相关知识的话,理解起来会相对容易一些,不过即使没有相关知识也没关系,书中的说明都很仔细,大家可以放心。

本书以IBM PC/AT兼容机(也就是所谓的Windows个人电脑)为对象进行说明。
\end{quote}

\section{	何谓操作系统	}
\section{	开发操作系统的各种方法	}
\section{	无知则无畏	}

\begin{quote}
当我们打算开发操作系统时,总会有人从旁边跳出来,罗列出一大堆专业术语,问这问那,像内核怎么做啦,外壳怎么做啦,是不是单片啦,是不是微内核啦,等等。虽然有时候提这些问题也是有益的,但一上来就问这些,当然会让人无从回答。

要想给他们一个满意答复,让他们不再从旁指手画脚的话,还真得多学习,拿出点像模像样的见解才行。但我们是初学者,没有必要去学那些麻烦的东西,费时费力且不说,当我们知道现有操作系统在各方面都考虑得如此周密的时候,就会发现自己的想法太过简单而备受打击没了干劲。如果被前人的成果吓倒,只用这些现有的技术来做些拼拼凑凑的工作,岂不是太没意思了。

所以我们这次不去学习那些复杂的东西,直接着手开发。就算知道一大堆专业术语、专业理论,又有什么意思呢?还不如动手去做,就算做出来的东西再简单,起码也是自己的成果。而且自己先实际操作一次,通过实践找到其中的问题,再来看看是不是已经有了这些问题的解决方案,这样下来更能深刻地理解那些复杂理论。不管怎么说,反正目前我们也无法回答那些五花八门的问题,倒不如直接告诉在一旁指手画脚的人们:我们就是想用自己的方法做自己喜欢的事情,如果要讨论高深的问题,就另请高明吧。
\end{quote}

作者苦口婆心地说了这么多就是希望如果你想开发个操作系统,就动手去写吧,到底自己重写个操作系统有什么用倒可以先放着。

如果你到现在还对要不要读这本书,或者读这本书的期望的收获有疑问,推荐你阅读豆瓣上本书的一篇评论\footnote{ http://book.douban.com/review/5606888/ }后再自行决定。
\begin{quote}
这本书对基础知识要求不高,懂点C语言和CPU基本知识就可以了,适合初学者。要是奔着了解操作系统原理或内核的期望,就不适宜读这本书了。30天后也许你真的可以向作者那样做出一个基本的系统模型,但这并不意味着你对内存管理、进程管理、设备管理有着怎样高深的认识。读这本书之前先弄清自己的定位吧,毕竟时间宝贵。
\end{quote}

\section{	如何开发操作系统	}
\section{	操作系统开发中的困难	}
\section{	学习本书时的注意事项(重要!)	}
\section{	各章内容摘要	}

