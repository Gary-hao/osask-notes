\chapter{	C语言与画面显示的练习	}
\section{	用C语言实现内存写入(harib01a)	}
在让画面黑屏的基础上,通过写VRAM的值,在画面上画出些“花”来。

\dag|projects\04_day\harib01a|
\begin{code}[label=naskfunc.nas]
_write_mem8:	; void write_mem8(int addr, int data);
		MOV		ECX,[ESP+4]		; [ESP+4]にaddrが入っているのでそれをECXに読み込む
		MOV		AL,[ESP+8]		; [ESP+8]にdataが入っているのでそれをALに読み込む
		MOV		[ECX],AL
		RET
\end{code}

C语言部分:
\begin{code}[label=bootpack.c]
void io_hlt(void);
void write_mem8(int addr, int data);

void HariMain(void)
{
	int i; /* 変数宣言。iという変数は、32ビットの整数型 */

	for (i = 0xa0000; i <= 0xaffff; i++) {
		write_mem8(i, 15); /* MOV BYTE [i],15 */
	}

	for (;;) {
		io_hlt();
	}
}
\end{code}

VRAM中都写入了15,意思是全部像素的颜色都是第15种颜色,即白色,因此运行程序后画面会变成白色。
\section{	条纹图案(harib01b)	}
\dag|projects\04_day\harib01b|
\begin{code}[label=bootpack.c]

	for (i = 0xa0000; i <= 0xaffff; i++) {
		write_mem8(i, i&0x0f);
	}
\end{code}

通过与运算,将15改成特殊的值,低4位保持不变,高4位全部变成0。这样每隔16个像素,色号就反复一次。


\section{	挑战指针(harib01c)	}
使用C语言的指针,修改上面程序,实现内存写入。

\dag|projects\04_day\harib01c|
\begin{code}[label=bootpack.c]
void io_hlt(void);

void HariMain(void)
{
	int i; /* 変数宣言。iという変数は、32ビットの整数型 */
	char *p; /* pという変数は、BYTE [...]用の番地 */

	for (i = 0xa0000; i <= 0xaffff; i++) {

		p = i; /* 番地を代入 */
		*p = i & 0x0f;

		/* これで write_mem8(i, i & 0x0f); の代わりになる */
	}

	for (;;) {
		io_hlt();
	}
}
\end{code}
\section{	指针的应用(1)(harib01d)	}
本节和下面一个小节做的事情和上面一样,只是换了种C语言的写法。

\dag|projects\04_day\harib01d|
\begin{code}[label=bootpack.c]
    p = (char *) 0xa0000; /* 番地を代入 */

	for (i = 0; i <= 0xffff; i++) {
		*(p + i) = i & 0x0f;
	}

\end{code}
\section{	指针的应用(2)(harib01e)	}
\dag|projects\04_day\harib01e|
\begin{code}[label=bootpack.c]
	p = (char *) 0xa0000; /* 番地を代入 */

	for (i = 0; i <= 0xffff; i++) {
		p[i] = i & 0x0f;
	}
\end{code}
\section{	色号设定(harib01f)	}
在描绘一个操作系统模样的画面之前,需要先处理颜色问题。这次使用的是320$\times$200的8位颜色模式,色号使用8位(二进制)数,也就是只能使用0$\sim$255的数。8位的彩色模式由程序员指定0$\sim$255的数字对应的颜色,比如25号颜色对应\#ffffff,26号颜色对应\#123456等。这种方式叫做调色板(palette)。

这里指定0$\sim$15这16种颜色。

\dag|projects\04_day\harib01f|
\begin{code}[label=bootpack.c]
void io_hlt(void);
void io_cli(void);
void io_out8(int port, int data);
int io_load_eflags(void);
void io_store_eflags(int eflags);

/* 実は同じソースファイルに書いてあっても、定義する前に使うのなら、
	やっぱり宣言しておかないといけない。 */

void init_palette(void);
void set_palette(int start, int end, unsigned char *rgb);

void HariMain(void)
{
	int i; /* 変数宣言。iという変数は、32ビットの整数型 */
	char *p; /* pという変数は、BYTE [...]用の番地 */

	init_palette(); /* パレットを設定 */

	p = (char *) 0xa0000; /* 番地を代入 */

	for (i = 0; i <= 0xffff; i++) {
		p[i] = i & 0x0f;
	}

	for (;;) {
		io_hlt();
	}
}

void init_palette(void)
{
	static unsigned char table_rgb[16 * 3] = {
		0x00, 0x00, 0x00,	/*  0:黒 */
		0xff, 0x00, 0x00,	/*  1:明るい赤 */
		0x00, 0xff, 0x00,	/*  2:明るい緑 */
		0xff, 0xff, 0x00,	/*  3:明るい黄色 */
		0x00, 0x00, 0xff,	/*  4:明るい青 */
		0xff, 0x00, 0xff,	/*  5:明るい紫 */
		0x00, 0xff, 0xff,	/*  6:明るい水色 */
		0xff, 0xff, 0xff,	/*  7:白 */
		0xc6, 0xc6, 0xc6,	/*  8:明るい灰色 */
		0x84, 0x00, 0x00,	/*  9:暗い赤 */
		0x00, 0x84, 0x00,	/* 10:暗い緑 */
		0x84, 0x84, 0x00,	/* 11:暗い黄色 */
		0x00, 0x00, 0x84,	/* 12:暗い青 */
		0x84, 0x00, 0x84,	/* 13:暗い紫 */
		0x00, 0x84, 0x84,	/* 14:暗い水色 */
		0x84, 0x84, 0x84	/* 15:暗い灰色 */
	};
	set_palette(0, 15, table_rgb);
	return;

	/* static char 命令は、データにしか使えないけどDB命令相当 */
}

void set_palette(int start, int end, unsigned char *rgb)
{
	int i, eflags;
	eflags = io_load_eflags();	/* 割り込み許可フラグの値を記録する */
	io_cli(); 					/* 許可フラグを0にして割り込み禁止にする */
	io_out8(0x03c8, start);
	for (i = start; i <= end; i++) {
		io_out8(0x03c9, rgb[0] / 4);
		io_out8(0x03c9, rgb[1] / 4);
		io_out8(0x03c9, rgb[2] / 4);
		rgb += 3;
	}
	io_store_eflags(eflags);	/* 割り込み許可フラグを元に戻す */
	return;
}
\end{code}

函数init\_palette开头一段以static开始的语句,声明了一个table\_rgb。

函数set\_palette中使用了0x03c8、0x03c9 之类的设备号码,这些设备号码来自于VGA的文档。

调色板的访问步骤:
\begin{itemize}
  \item 首先在一连串的访问中屏蔽中断(比如CLI)。
  \item 将想要设定的调色板号码写入0x03c8,紧接着,按R,G,B的顺序写入0x03c9。如果还想继续设定下一个调色板,则省略调色板号码,再按照RGB的顺序写入0x03c9即可。
  \item 如果想要读出当前调色板的状态,首先要将调色板的号码写入0x03c7,再从0x03c9读取三次。读出的顺序就是R,G,B。如果要继续读出下一个调色板,同样是省略调色板号码,按RGB的顺序读出。
  \item 如果最初执行了CLI,那么最后要执行STI。
\end{itemize}

这里提到的CLI(clear interrupt flag)是将中断标志置为0的指令,STI(set interrupt flag)是将中断标志置为1的指令。

set\_palette中想要做的事情是在设定调色板之前首先执行CLI,处理结束后恢复中断标志,通过谢了函数io\_load\_eflags来读取最初的eflags 值,处理结束后直接用io\_store\_eflags恢复。

\begin{code}[label=naskfunc.nas]
; naskfunc
; TAB=4

[FORMAT "WCOFF"]				; オブジェクトファイルを作るモード	
[INSTRSET "i486p"]				; 486の命令まで使いたいという記述
[BITS 32]						; 32ビットモード用の機械語を作らせる
[FILE "naskfunc.nas"]			; ソースファイル名情報

		GLOBAL	_io_hlt, _io_cli, _io_sti, _io_stihlt
		GLOBAL	_io_in8,  _io_in16,  _io_in32
		GLOBAL	_io_out8, _io_out16, _io_out32
		GLOBAL	_io_load_eflags, _io_store_eflags

[SECTION .text]

_io_hlt:	; void io_hlt(void);
		HLT
		RET

_io_cli:	; void io_cli(void);
		CLI
		RET

_io_sti:	; void io_sti(void);
		STI
		RET

_io_stihlt:	; void io_stihlt(void);
		STI
		HLT
		RET

_io_in8:	; int io_in8(int port);
		MOV		EDX,[ESP+4]		; port
		MOV		EAX,0
		IN		AL,DX
		RET

_io_in16:	; int io_in16(int port);
		MOV		EDX,[ESP+4]		; port
		MOV		EAX,0
		IN		AX,DX
		RET

_io_in32:	; int io_in32(int port);
		MOV		EDX,[ESP+4]		; port
		IN		EAX,DX
		RET

_io_out8:	; void io_out8(int port, int data);
		MOV		EDX,[ESP+4]		; port
		MOV		AL,[ESP+8]		; data
		OUT		DX,AL
		RET

_io_out16:	; void io_out16(int port, int data);
		MOV		EDX,[ESP+4]		; port
		MOV		EAX,[ESP+8]		; data
		OUT		DX,AX
		RET

_io_out32:	; void io_out32(int port, int data);
		MOV		EDX,[ESP+4]		; port
		MOV		EAX,[ESP+8]		; data
		OUT		DX,EAX
		RET

_io_load_eflags:	; int io_load_eflags(void);
		PUSHFD		; PUSH EFLAGS という意味
		POP		EAX
		RET

_io_store_eflags:	; void io_store_eflags(int eflags);
		MOV		EAX,[ESP+4]
		PUSH	EAX
		POPFD		; POP EFLAGS という意味
		RET
\end{code}

程序中,PUSHFD是“push flags double-world”的缩写,POPFD是“pop flags double-world”的缩写。
\section{	绘制矩形(harib01g)	}
开始绘制一些图形。

首先从VRAM与画面上的“点”的关系开始说起。

在当前画面模式中,画面上有320$\times$200(64000)个像素。假设左上点的坐标是(0,0),右下点的坐标是(319319),那么像素坐标(x,y)对应的VRAM地址按下式计算:
\begin{equation*}
  0xa0000+x+y*320
\end{equation*}
按照上式计算像素的地址,往该地址的内存里存放某种颜色的号码,那么画面上该像素的位置就出现相应的颜色。这样就画了一个点。继续增加x的值,循环以上操作,就能画一条直线,再向下循环这条直线,就能画出很多直线,组成一个有填充色的长方形。

\dag|projects\04_day\harib01g|
\begin{code}[label=bootpack.c]
#define COL8_000000		0
#define COL8_FF0000		1
#define COL8_00FF00		2
#define COL8_FFFF00		3
#define COL8_0000FF		4
#define COL8_FF00FF		5
#define COL8_00FFFF		6
#define COL8_FFFFFF		7
#define COL8_C6C6C6		8
#define COL8_840000		9
#define COL8_008400		10
#define COL8_848400		11
#define COL8_000084		12
#define COL8_840084		13
#define COL8_008484		14
#define COL8_848484		15

void HariMain(void)
{
	char *p; /* pという変数は、BYTE [...]用の番地 */

	init_palette(); /* パレットを設定 */

	p = (char *) 0xa0000; /* 番地を代入 */

	boxfill8(p, 320, COL8_FF0000,  20,  20, 120, 120);
	boxfill8(p, 320, COL8_00FF00,  70,  50, 170, 150);
	boxfill8(p, 320, COL8_0000FF, 120,  80, 220, 180);

	for (;;) {
		io_hlt();
	}
}

void boxfill8(unsigned char *vram, int xsize, unsigned char c, int x0, int y0, int x1, int y1)
{
	int x, y;
	for (y = y0; y <= y1; y++) {
		for (x = x0; x <= x1; x++)
			vram[y * xsize + x] = c;
	}
	return;
}
\end{code}

上面的程序写了一个boxfill8的函数,被调用3次,绘制了3个矩形。
\section{	今天的成果(harib01h)	}
绘制一个简单的带任务条的系统桌面,简单到简陋……竟然有凹凸效果!

\dag|projects\04_day\harib01h|
\begin{code}[label=bootpack.c]
void HariMain(void)
{
	char *vram;
	int xsize, ysize;

	init_palette();
	vram = (char *) 0xa0000;
	xsize = 320;
	ysize = 200;

	boxfill8(vram, xsize, COL8_008484,  0,         0,          xsize -  1, ysize - 29);
	boxfill8(vram, xsize, COL8_C6C6C6,  0,         ysize - 28, xsize -  1, ysize - 28);
	boxfill8(vram, xsize, COL8_FFFFFF,  0,         ysize - 27, xsize -  1, ysize - 27);
	boxfill8(vram, xsize, COL8_C6C6C6,  0,         ysize - 26, xsize -  1, ysize -  1);

	boxfill8(vram, xsize, COL8_FFFFFF,  3,         ysize - 24, 59,         ysize - 24);
	boxfill8(vram, xsize, COL8_FFFFFF,  2,         ysize - 24,  2,         ysize -  4);
	boxfill8(vram, xsize, COL8_848484,  3,         ysize -  4, 59,         ysize -  4);
	boxfill8(vram, xsize, COL8_848484, 59,         ysize - 23, 59,         ysize -  5);
	boxfill8(vram, xsize, COL8_000000,  2,         ysize -  3, 59,         ysize -  3);
	boxfill8(vram, xsize, COL8_000000, 60,         ysize - 24, 60,         ysize -  3);

	boxfill8(vram, xsize, COL8_848484, xsize - 47, ysize - 24, xsize -  4, ysize - 24);
	boxfill8(vram, xsize, COL8_848484, xsize - 47, ysize - 23, xsize - 47, ysize -  4);
	boxfill8(vram, xsize, COL8_FFFFFF, xsize - 47, ysize -  3, xsize -  4, ysize -  3);
	boxfill8(vram, xsize, COL8_FFFFFF, xsize -  3, ysize - 24, xsize -  3, ysize -  3);

	for (;;) {
		io_hlt();
	}
}
\end{code}


