\chapter{	叠加处理	}
\section{	内存管理(续)(harib07a)	}

为了以后使用起来更加方便,我们还是把这些内存管理函数再整理一下。|memman_alloc|和|memman_free|能够以1字节为单位进行内存管理,这种方式虽然不错,但是有一点不足——在反复进行内存分配和内存释放之后,内存中就会出现很多不连续的小段未使用空间,这样就会把|man—>frees|消耗殆尽。

编写一些总是以0x1000字节为单位进行内存分配和释放的函数,它们会把指定的内存大小按0x1000字节为单位向上舍入(roundup),0x1000字节的大小正好是4KB。

\begin{code}[label=memory.c节选]
unsigned int memman_alloc_4k(struct MEMMAN *man, unsigned int size)
{
	unsigned int a;
	size = (size + 0xfff) & 0xfffff000;
	a = memman_alloc(man, size);
	return a;
}

int memman_free_4k(struct MEMMAN *man, unsigned int addr, unsigned int size)
{
	int i;
	size = (size + 0xfff) & 0xfffff000;
	i = memman_free(man, addr, size);
	return i;
}
\end{code}

增加的部分使用了向上舍入。|i = (i + 0xfff) & 0xfffff000;|是“加上0xfff后进行向下舍入”的运算。

由于十六进制不易理解,所以以十进制运算为例来说明。如使用这个方法以100 为单位对456进行向上舍入,就相当于先加上99再进行向下舍入。456 加上99是555,向下舍入后就是500了。那么如果对400进行向上舍入呢?先加上99,得到499,再进行向下舍入,结果是400。 看,400向上舍入的结果还是400。
\section{	叠加处理(harib07b)	}
在画面上进行叠加显示,类似于将绘制了图案的透明图层叠加在一起。

实际上,我们并不是仅仅把两张大小相同的图层重叠在一起,而是要从大到小准备很多张图层。

最上面的小图层用来描绘鼠标指针,它下面的几张图层是用来存放窗口的,而最下面的一张图层用来存放桌面壁纸。同时,我们还要通过移动图层的方法实现鼠标指针的移动以及窗口的移动。

\cs

\begin{code}
struct SHEET {
    unsigned char *buf;
    int bxsize, bysize, vx0, vy0, col_inv, height, flags;
}
\end{code}

暂时先写成这样就可以了。程序里的sheet这个词,表示“透明图层”的意思。笔者觉得英文里没有和“透明图层”接近的词,就凭感觉选了它。buf 是用来记录图层上所描画内容的地址(buffer的略语)。图层的整体大小,用bxsize*bysize表示。vx0和vy0是表示图层在画面上位置的坐标,v 是VRAM的略语。|col_inv|表示透明色色号,它是color (颜色)和invisible(透明)的组合略语。height表示图层高度。flags用于存放有关图层的各种设定信息。

创建一个管理多重图层信息的结构。

\begin{code}
#define MAX_SHEETS      256

struct SHTCTL {
    unsigned char *vram;
    int xsize, ysize, top;
    struct SHEET *sheets[MAX_SHEETS];
    struct SHEET sheets0[MAX_SHEETS];
};
\end{code}

创建了SHTCTL结构体,其名称来源于sheet control的略语,意思是“图层管理”。 |MAX_SHEETS|是能够管理的最大图层数,这个值设为256 应该够用了。

变量vram、xsize、ysize代表VRAM的地址和画面的大小,但如果每次都从BOOTINFO查询的话就太麻烦了,所以在这里预先对它们进行赋值操作。top代表最上面图层的高度。sheets0这个结构体用于存放准备的256个图层的信息。而sheets是记忆地址变量的领域,所以相应地也要先准备256 份。由于sheets0中的图层顺序混乱,所以把它们按照高度进行升序排列,然后将其地址写入sheets中,这样就方便多了。

\cs

\begin{code}
struct SHTCTL *shtctl_init(struct MEMMAN *memman, unsigned char *vram, int xsize, int ysize)
{
	struct SHTCTL *ctl;
	int i;
	ctl = (struct SHTCTL *) memman_alloc_4k(memman, sizeof (struct SHTCTL));
	if (ctl == 0) {
		goto err;
	}
	ctl->vram = vram;
	ctl->xsize = xsize;
	ctl->ysize = ysize;
	ctl->top = -1; /* シートは一枚もない */
	for (i = 0; i < MAX_SHEETS; i++) {
		ctl->sheets0[i].flags = 0; /* 未使用マーク */
	}
err:
	return ctl;
}
\end{code}

程序首先使用|memman_alloc_4k|来分配用于记忆图层控制变量的内存空间

接着,给控制变量赋值,给其下的所有图层变量都加上“未使用”标签。做完这一步,这个函数就完成了。

\cs

再做一个函数,用于取得新生成的未使用图层。

\begin{code}
#define SHEET_USE		1

struct SHEET *sheet_alloc(struct SHTCTL *ctl)
{
	struct SHEET *sht;
	int i;
	for (i = 0; i < MAX_SHEETS; i++) {
		if (ctl->sheets0[i].flags == 0) {
			sht = &ctl->sheets0[i];
			sht->flags = SHEET_USE; /* 使用中マーク */
			sht->height = -1; /* 非表示中 */
			return sht;
		}
	}
	return 0;	/* 全てのシートが使用中だった */
}
\end{code}

在|sheets0[]|中寻找未使用的图层,如果找到了,就将其标记为“正在使用”,并返回其地址就可以了,这里没有什么难点。高度设为-1,表示图层的高度还没有设置,因而不是显示对象。

\cs

\begin{code}
void sheet_setbuf(struct SHEET *sht, unsigned char *buf, int xsize, int ysize, int col_inv)
{
	sht->buf = buf;
	sht->bxsize = xsize;
	sht->bysize = ysize;
	sht->col_inv = col_inv;
	return;
}
\end{code}

这是设定图层的缓冲区大小和透明色的函数。

\cs

写设定底板高度的函数。
\begin{code}
void sheet_updown(struct SHTCTL *ctl, struct SHEET *sht, int height)
{
	int h, old = sht->height; /* 設定前の高さを記憶する */

	/* 指定が低すぎや高すぎだったら、修正する */
	if (height > ctl->top + 1) {
		height = ctl->top + 1;
	}
	if (height < -1) {
		height = -1;
	}
	sht->height = height; /* 高さを設定 */

	/* 以下は主にsheets[]の並べ替え */
	if (old > height) {	/* 以前よりも低くなる */
		if (height >= 0) {
			/* 間のものを引き上げる */
			for (h = old; h > height; h--) {
				ctl->sheets[h] = ctl->sheets[h - 1];
				ctl->sheets[h]->height = h;
			}
			ctl->sheets[height] = sht;
		} else {	/* 非表示化 */
			if (ctl->top > old) {
				/* 上になっているものをおろす */
				for (h = old; h < ctl->top; h++) {
					ctl->sheets[h] = ctl->sheets[h + 1];
					ctl->sheets[h]->height = h;
				}
			}
			ctl->top--; /* 表示中の下じきが一つ減るので、一番上の高さが減る */
		}
		sheet_refresh(ctl); /* 新しい下じきの情報に沿って画面を描き直す */
	} else if (old < height) {	/* 以前よりも高くなる */
		if (old >= 0) {
			/* 間のものを押し下げる */
			for (h = old; h < height; h++) {
				ctl->sheets[h] = ctl->sheets[h + 1];
				ctl->sheets[h]->height = h;
			}
			ctl->sheets[height] = sht;
		} else {	/* 非表示状態から表示状態へ */
			/* 上になるものを持ち上げる */
			for (h = ctl->top; h >= height; h--) {
				ctl->sheets[h + 1] = ctl->sheets[h];
				ctl->sheets[h + 1]->height = h + 1;
			}
			ctl->sheets[height] = sht;
			ctl->top++; /* 表示中の下じきが一つ増えるので、一番上の高さが増える */
		}
		sheet_refresh(ctl); /* 新しい下じきの情報に沿って画面を描き直す */
	}
	return;
}
\end{code}

\cs

在|sheet_updown|中使用了|sheet_refresh|函数,这个函数会从下到上描绘所有的图层。

\begin{code}
void sheet_refresh(struct SHTCTL *ctl)
{
	int h, bx, by, vx, vy;
	unsigned char *buf, c, *vram = ctl->vram;
	struct SHEET *sht;
	for (h = 0; h <= ctl->top; h++) {
		sht = ctl->sheets[h];
		buf = sht->buf;
		for (by = 0; by < sht->bysize; by++) {
			vy = sht->vy0 + by;
			for (bx = 0; bx < sht->bxsize; bx++) {
				vx = sht->vx0 + bx;
				c = buf[by * sht->bxsize + bx];
				if (c != sht->col_inv) {
					vram[vy * ctl->xsize + vx] = c;
				}
			}
		}
	}
	return;
}
\end{code}

对于已设定了高度的所有图层而言,要从下往上,将透明以外的所有像素都复制到VRAM中。由于是从下开始复制,所以最后最上面的内容就留在了画面上。

\cs

不改变图层高度而只上下左右移动图层的函数——|sheet_slide|。

\begin{code}
void sheet_slide(struct SHTCTL *ctl, struct SHEET *sht, int vx0, int vy0)
{
	sht->vx0 = vx0;
	sht->vy0 = vy0;
	if (sht->height >= 0) { /* もしも表示中なら */
		sheet_refresh(ctl); /* 新しい下じきの情報に沿って画面を描き直す */
	}
	return;
}
\end{code}

最后是释放已使用图层的内存的函数|sheet_free|。

\begin{code}
void sheet_free(struct SHTCTL *ctl, struct SHEET *sht)
{
	if (sht->height >= 0) {
		sheet_updown(ctl, sht, -1); /* 表示中ならまず非表示にする */
	}
	sht->flags = 0; /* 未使用マーク */
	return;
}
\end{code}

\cs
将以上与图层相关的程序汇总到sheet.c中,所以就要改造HariMain函数了。

\begin{code}
void HariMain(void)
{
    (中略)
	struct SHTCTL *shtctl;
	struct SHEET *sht_back, *sht_mouse;
	unsigned char *buf_back, buf_mouse[256];

    (中略)

	init_palette();
	shtctl = shtctl_init(memman, binfo->vram, binfo->scrnx, binfo->scrny);
	sht_back  = sheet_alloc(shtctl);
	sht_mouse = sheet_alloc(shtctl);
	buf_back  = (unsigned char *) memman_alloc_4k(memman, binfo->scrnx * binfo->scrny);
	sheet_setbuf(sht_back, buf_back, binfo->scrnx, binfo->scrny, -1); /* 透明色なし */
	sheet_setbuf(sht_mouse, buf_mouse, 16, 16, 99);
	init_screen8(buf_back, binfo->scrnx, binfo->scrny);
	init_mouse_cursor8(buf_mouse, 99);
	sheet_slide(shtctl, sht_back, 0, 0);
	mx = (binfo->scrnx - 16) / 2; /* 画面中央になるように座標計算 */
	my = (binfo->scrny - 28 - 16) / 2;
	sheet_slide(shtctl, sht_mouse, mx, my);
	sheet_updown(shtctl, sht_back,  0);
	sheet_updown(shtctl, sht_mouse, 1);
	sprintf(s, "(%3d, %3d)", mx, my);
	putfonts8_asc(buf_back, binfo->scrnx, 0, 0, COL8_FFFFFF, s);
	sprintf(s, "memory %dMB   free : %dKB",
			memtotal / (1024 * 1024), memman_total(memman) / 1024);
	putfonts8_asc(buf_back, binfo->scrnx, 0, 32, COL8_FFFFFF, s);
	sheet_refresh(shtctl);

	for (;;) {
		io_cli();
		if (fifo8_status(&keyfifo) + fifo8_status(&mousefifo) == 0) {
			io_stihlt();
		} else {
			if (fifo8_status(&keyfifo) != 0) {
				i = fifo8_get(&keyfifo);
				io_sti();
				sprintf(s, "%02X", i);
				boxfill8(buf_back, binfo->scrnx, COL8_008484,  0, 16, 15, 31);
				putfonts8_asc(buf_back, binfo->scrnx, 0, 16, COL8_FFFFFF, s);
				sheet_refresh(shtctl);
			} else if (fifo8_status(&mousefifo) != 0) {
				i = fifo8_get(&mousefifo);
				io_sti();
				if (mouse_decode(&mdec, i) != 0) {
					/* データが3バイト揃ったので表示 */
					sprintf(s, "[lcr %4d %4d]", mdec.x, mdec.y);
					if ((mdec.btn & 0x01) != 0) {
						s[1] = 'L';
					}
					if ((mdec.btn & 0x02) != 0) {
						s[3] = 'R';
					}
					if ((mdec.btn & 0x04) != 0) {
						s[2] = 'C';
					}
					boxfill8(buf_back, binfo->scrnx, COL8_008484, 32, 16, 32 + 15 * 8 - 1, 31);
					putfonts8_asc(buf_back, binfo->scrnx, 32, 16, COL8_FFFFFF, s);
					/* マウスカーソルの移動 */
					mx += mdec.x;
					my += mdec.y;
					if (mx < 0) {
						mx = 0;
					}
					if (my < 0) {
						my = 0;
					}
					if (mx > binfo->scrnx - 16) {
						mx = binfo->scrnx - 16;
					}
					if (my > binfo->scrny - 16) {
						my = binfo->scrny - 16;
					}
					sprintf(s, "(%3d, %3d)", mx, my);
					boxfill8(buf_back, binfo->scrnx, COL8_008484, 0, 0, 79, 15); /* 座標消す */
					putfonts8_asc(buf_back, binfo->scrnx, 0, 0, COL8_FFFFFF, s); /* 座標書く */
					sheet_slide(shtctl, sht_mouse, mx, my); /* sheet_refreshを含む */
				}
			}
		}
	}
}
\end{code}

我们准备了2个图层,分别是|sht_back|和|sht_mouse|,还准备了2个缓冲区|buf_back| 和|buf_mouse|,用于在其中描绘图形。以前我们指定为|binfo —> vram| 的部分,现在有很多都改成了|buf_back|。

运行。

\section{	提高叠加处理速度(1)(harib07c)	}
那么怎样才能提高速度呢?首先,我们从鼠标指针的移动,也就是图层的移动来思考一下。

鼠标指针虽然最多只有16×16=256个像素,可根据harib07b的原理,只要它稍一移动,程序就会对整个画面进行刷新,也就是重新描绘320×200=64 000个像素。而实际上,只重新描绘移动相关的部分,也就是移动前后的部分就可以了,即256×2=512个像素。这只是64 000像素的0.8\%而已,所以有望提速很多。现在我们根据这个思路写一下程序。

\cs

\begin{code}
void sheet_refreshsub(struct SHTCTL *ctl, int vx0, int vy0, int vx1, int vy1)
{
    int h, bx, by, vx, vy;
    unsigned char *buf, c, *vram = ctl->vram;
    struct SHEET *sht;
    for (h = 0; h <= ctl->top; h++) {
        sht = ctl->sheets[h];
        buf = sht->buf;
        for (by = 0; by < sht->bysize; by++) {
            vy = sht->vy0 + by;
            for (bx = 0; bx < sht->bxsize; bx++) {
                vx = sht->vx0 + bx;
                if (vx0 <= vx && vx < vx1 && vy0 <= vy && vy < vy1) {
                    c = buf[by * sht->bxsize + bx];
                    if (c != sht->col_inv) {
                        vram[vy * ctl->xsize + vx] = c;
                    }
                }
            }
        }
    }
    return;
}
\end{code}

这个函数几乎和|sheet_refresh|一样,唯一的不同点在于它能使用vx0$\sim$vy1 指定刷新的范围。

\cs

现在我们使用这个refreshsub函数来提高|sheet_slide|的运行速度。

\begin{code}
void sheet_slide(struct SHTCTL *ctl, struct SHEET *sht, int vx0, int vy0)
{
	int old_vx0 = sht->vx0, old_vy0 = sht->vy0;
	sht->vx0 = vx0;
	sht->vy0 = vy0;
	if (sht->height >= 0) { /* もしも表示中なら、新しい下じきの情報に沿って画面を描き直す */
		sheet_refreshsub(ctl, old_vx0, old_vy0, old_vx0 + sht->bxsize, old_vy0 + sht->bysize);
		sheet_refreshsub(ctl, vx0, vy0, vx0 + sht->bxsize, vy0 + sht->bysize);
	}
	return;
}
\end{code}

这段程序所做的是:首先记住移动前的显示位置,再设定新的显示位置,最后只要重新描绘移动前和移动后的地方。

\cs

我们所说的在图层上显示文字,实际上并不是改写图层的全部内容。假设我们已经写了20个字,那么8×16×20=2560,也就是仅仅重写2560个像素的内容就应该足够了。但现在每次却要重写64 000个像素的内容,所以速度才那么慢。

这么说来,这里好像也可以使用refreshsub,那么我们就来重新编写函数|sheet_refresh|吧。

\begin{code}
void sheet_refresh(struct SHTCTL *ctl, struct SHEET *sht, int bx0, int by0, int bx1, int by1)
{
	if (sht->height >= 0) { /* もしも表示中なら、新しい下じきの情報に沿って画面を描き直す */
		sheet_refreshsub(ctl, sht->vx0 + bx0, sht->vy0 + by0, sht->vx0 + bx1, sht->vy0 + by1);
	}
	return;
}
\end{code}

所谓指定范围,并不是直接指定画面内的坐标,而是以缓冲区内的坐标来表示。这样一来,HariMain就可以不考虑图层在画面中的位置了。

相应修改|sheet_updown|函数。做了改动的只有|sheet_refresh|(ctl)这部分(有两处)。

改写HariMain,改写了其中的|sheet_refresh|,变更点共有4个。只有每次要往|buf_back| 中写入信息时,才进行|sheet_refresh|。


\section{	提高叠加处理速度(2)(harib07d)	}
|sheet_refreshsub| 即使不写入像素内容,也要多次执行if语句,这一点不太好,如果能改善一下,速度应该会提高不少。

按照上面这种写法,即便只刷新图层的一部分,也要对所有图层的全部像素执行if语句,判断“是写入呢,还是不写呢”。而对于刷新范围以外的部分,就算执行if判断语句,最后也不会进行刷新,所以这纯粹就是一种浪费。既然如此,最初就应该把for语句的范围限定在刷新范围之内。

\begin{code}
void sheet_refreshsub(struct SHTCTL *ctl, int vx0, int vy0, int vx1, int vy1)
{
	int h, bx, by, vx, vy, bx0, by0, bx1, by1;
	unsigned char *buf, c, *vram = ctl->vram;
	struct SHEET *sht;
	for (h = 0; h <= ctl->top; h++) {
		sht = ctl->sheets[h];
		buf = sht->buf;
		/* vx0~vy1を使って、bx0~by1を逆算する */
		bx0 = vx0 - sht->vx0;
		by0 = vy0 - sht->vy0;
		bx1 = vx1 - sht->vx0;
		by1 = vy1 - sht->vy0;
		if (bx0 < 0) { bx0 = 0; }
		if (by0 < 0) { by0 = 0; }
		if (bx1 > sht->bxsize) { bx1 = sht->bxsize; }
		if (by1 > sht->bysize) { by1 = sht->bysize; }
		for (by = by0; by < by1; by++) {
			vy = sht->vy0 + by;
			for (bx = bx0; bx < bx1; bx++) {
				vx = sht->vx0 + bx;
				c = buf[by * sht->bxsize + bx];
				if (c != sht->col_inv) {
					vram[vy * ctl->xsize + vx] = c;
				}
			}
		}
	}
	return;
}
\end{code}

改良的关键在于,bx在for语句中并不是在0到bxsize之间循环,而是在bx0到bx1之间循环(对于by也一样)。而bx0和bx1都是从刷新范围“倒推”求得的。倒推其实就是把公式变形转换了一下,计算vx0的坐标相当于bx中的哪个位置,然后把它作为bx0。其他的坐标处理方法也一样。

运行一下!